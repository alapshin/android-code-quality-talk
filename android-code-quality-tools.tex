\documentclass{beamer}

\usetheme{metropolis}
\usefonttheme{serif}
\setbeameroption{show notes}
\setbeamercolor{note page}{bg=white}
\setbeamercolor{note title}{bg=white}
\setbeamertemplate{bibliography item}{}

\usepackage{fontspec}
\setmainfont{Noto Serif}
\setsansfont{Noto Sans}
\setmonofont{Noto Mono}
\newfontfamily{\cyrillicfont}{Noto Serif}
\newfontfamily{\cyrillicfonttt}{Noto Mono}
\usepackage{microtype}

\usepackage{polyglossia}
\setmainlanguage{english}
\setotherlanguage{russian}

\usepackage{csquotes}
\usepackage{graphicx}
\usepackage{ragged2e}
\usepackage[sorting=none]{biblatex}

% Remove last dot from biblatex entries
\renewcommand{\finentrypunct}{}
% Replace dot between title and url with colon
\renewcommand{\newunitpunct}{\addcolon\space}
% Remove URL prefix from url
\DeclareFieldFormat{url}{\url{#1}}
\addbibresource{android-code-quality-tools.bib}

\usepackage{hyperref}
\hypersetup{breaklinks=true,unicode=true,pdfencoding=auto}

\usepackage{shellesc}
\usepackage[outputdir=build]{minted}

\author{Андрей Лапшин}
\date{2017-12-26}
\title{Инструменты контроля качества кода}


\begin{document}
\begin{frame}
    \titlepage
    \note{
        Всем привет. Меня зовут Андрей Лапшин и мой доклад посвящен
        различный инструментам контроля качества кода в Android-проектах.
    }
\end{frame}

\begin{frame}
    \frametitle{Зачем?}
    \pause
    \begin{itemize}
        \item{Корректность}
        \pause
        \item{Производительность}
        \pause
        \item{Простота поддержки}
    \end{itemize}
    \note<1>{
        Прежде чем рассматривать конкретные инструменты нужно определиться,
        для чего они необходимы.
    }
    \note<2>{
        Во-первых, при написании кода мы хотим чтобы он не содержал ошибок.
        Но как мы можем это гарантировать?
        Как вариант, мы можем внедрить тестирование (ручное и/или автоматическое)
        Но ручное тестирование и написание тестов требует времени и как правило
        сосредоточено на тестировании бизнес-логики. В тоже время практически
        любой язык программирования cодержит набор шаблонов, про которые известно,
        что они могут привести к ошибкам. Например сравнение строк. Было бы
        хорошо, чтобы такие ситуации отлавливались автоматический и как можно раньше.
    }
    \note<3>{
        Во-вторых, мы хотим, чтобы написанный код был производительным.
        Но как мы можем найти проблемные места, которые ведут к снижению
        производительности? Возможно в данный момент эти места не так очевидны.
        В тоже время есть известные шаблоны, которые ведут к снижению
        производительности, например выделение памяти в \texttt{onDraw}
    }
    \note<4>{
        В-третьих, даже если написанный код корректен и производителен возникает
        вопрос о простоте поддержки этого кода в будущем. Возможно этот код
        написан в одном файле размеров несколько тысяч строк и при этом ещё
        части этого файла используют разные стили оформления в разных частях.
    }
    \note<4>{
        К сожалению, ручной поиск перечисленных проблем не является реальным
        подходом.  Во-первых, маловероятно, что каждый разработчик знает все
        подводные камни языка. Во-вторых, даже если разработчик знает все
        возможные проблемы все равно есть вероятность что-то пропустить. Поэтому
        поиск подобных проблем должен быть автоматизирован.
    }
\end{frame}

\begin{frame}
    \frametitle{Android Studio (и другие IDE от JetBrains)}
    Набор проверок поставляемый в комплекте с IDE
    \begin{block}{Возможности}
        \begin{itemize}
            \item{Потенциальные ошибки}
            \item{Проблемы производительности}
            \item{Поиск неиспользуемого кода}
            \item{Проверка кода на сложность}
            \item{и т. д.}
        \end{itemize}
    \end{block}
    \begin{block}{Запуск}
        Пункт главного меню \texttt{Analyze | Inspect Code}
    \end{block}
    \note{
        Каждая IDE от JetBrains (Android Studio в том числе) поставляется с
        набором проверок, которые можно запустить из главного меню.
    }
\end{frame}
\begin{frame}
    \frametitle{Android Studio (и другие IDE от JetBrains)}
    \begin{block}{Плюсы}
        \begin{itemize}
            \item{Отличная интеграция в IDE}
            \item{Поиск проблем на лету}
            \item{Возможность автоматического исправления найденных проблем}
            \item{Большой набор проверок для разных языков}
        \end{itemize}
    \end{block}
    \begin{block}{Минусы}
        \begin{itemize}
            \item{Сложность интеграции в автоматизированный процесс сборки}
        \end{itemize}
    \end{block}
    \note{
        Так как эти проверки интегрированный в IDE, то их удобно использовать
        при локальной разработке. Плюс IDE автоматический подсвечивает проблемный
        код и предлагает варианты исправления. Кроме того проверки не ограничены
        только Java или Kotlin. Например Android Studio содержит проверки для
        C/C++ кода и дополнительный проверки специфичные для Android-проектов.

        К сожалению, проверки из Intellij IDEA сложно интегрировать в автоматизированный
        процесс сборки на CI. Intellij IDEA позволяет запускать проверки из командной строки, но
        этого не достаточно для полноценной интеграции.
        Во-первых, нужно устанавливать на CI копию IDE.
        Во-вторых, одновременно может быть запущена только одна копия IDE, т. е.
        невозможно одновременно выполнять > 1 билда.
        В-третьих, при запуске из командой строки отчет о найденных ошибках выдается в
        виде нескольких XML-файлов, а не HTML. TeamCity (CI от JetBrains) умеет показывать
        эти отчеты в виде пригодном для чтения, однако для других CI такой поддержки нет.
        Как вариант, можно преобразовывать эти XML-файлы в HTML используя XSLT.
    }
\end{frame}

\begin{frame}
    \frametitle{Checkstyle}
    Статический анализатор для проверки кода на соответствие стилю.
    \begin{block}{Возможности}
        \begin{itemize}
            \item{Проверка отступов, пробелов}
            \item{Наименование переменных}
            \item{Ограничение на длину строк}
            \item{Определение неиспользуемых импортов}
            \item{и т. д.}
        \end{itemize}
    \end{block}
    \begin{block}{Запуск}
        Gradle-таск и/или плагин IDE
    \end{block}
    \note{
        Следующий инструмент, который полезен для поддержания качества кода это
        Checkstyle. В первую очередь он предназначен для проверки кода на соответствие
        определенному стилю (размер отступов, расстановка скобок и т. п.).

        Checkstyle-плагин идет в стандартной поставке Gradle, но для Android
        проектов может потребоваться небольшая конфигурация. Как вариант есть
        плагины для Android Studio, который позволяют запускать Checkstyle из IDE.
    }
\end{frame}

\begin{frame}
    \frametitle{Checkstyle}
    \begin{block}{Плюсы}
        \begin{itemize}
            \item{Скорость проверки}
            \item{Гибкие возможности настройки}
            \item{Возможность интеграции в автоматизированный процесс сборки}
        \end{itemize}
    \end{block}
    \begin{block}{Минусы}
        \begin{itemize}
            \item{Требует конфигурации}
            \item{Сложность интеграции в IDE}
        \end{itemize}
    \end{block}
    \note{
        К плюсам Checkstyle можно отнести скорость проверки (так как не требует
        компиляции проекта, то может выполняться в качестве первого шага сборки)
        Также Checkstyle позволяет очень гибко настроить стиль за счет большого
        набора опций. И наконец Checkstyle легко интегрировать в процесс сборки.
        Так как это стандартный Gradle-плагин, то есть возможность запускать его
        из командной строки и останавливать билд при наличии ошибок, а результаты
        можно получать либо в виде HTML, либо в виде XML.

        К минусам можно отнести необходимость дополнительной конфигурации (создание
        конфигурационного файла и Gradle-таска) и некоторую сложность использования
        в IDE, так как требуется установка дополнительного плагина.
    }
\end{frame}

\begin{frame}
    \frametitle{FindBugs/SpotBugs}
    Статический анализатор для поиска потенциальных ошибок. Проверяет байткод.
    \begin{block}{Возможности}
        \begin{itemize}
            \item{Проверка кода на корректность}
            \item{Проверка на использование известных проблемных подходов}
            \item{Проверка на производительность}
            \item{и т. д.}
        \end{itemize}
    \end{block}
    \begin{block}{Запуск}
        Gradle-таск и/или плагин IDE
    \end{block}
    \note{
        Ещё один статический анализатор доступный для использования. В первую
        очередь ориентирован на проверку кода на корректность. Например, может
        найти приведение типа, которое всегда ведет к \texttt{ClassCastException}
        Помимо проверки на корректность, может находить использование плохих практик.
        Например использование оператора \texttt{==} вместо \texttt{equals()} для
        сравнения строк.

        Также как и Checkstyle Findbugs-плагин идет в комплекте с Gradle, но
        требует небольшой дополнительной конфигурации. Как вариант есть плагины
        для Android Studio, который позволяют запускать FindBugs из IDE.

        Для информации: оригинальный FindBugs проект находится в полумертвом
        состоянии, поэтому некоторые из разработчиков создали форк SpotBugs.
    }
\end{frame}

\begin{frame}
    \frametitle{FindBugs/SpotBugs}
    \begin{block}{Плюсы}
        \begin{itemize}
            \item{Высокая точность}
            \item{Возможность интеграции в автоматизированный процесс сборки}
        \end{itemize}
    \end{block}
    \begin{block}{Минусы}
        \begin{itemize}
            \item{Требует конфигурации}
            \item{Сложность интеграции в IDE}
            \item{Возможны ложные срабатывания в Kotlin коде}
            \item{Временные и аппаратные требования}
        \end{itemize}
    \end{block}
    \note{
        К плюсам FindBugs можно отнести высокую точность. Найденные проблемы как
        правило действительно являются проблемами, которые нужно исправлять.
        Наличие Gradle-плагина позволяет легко интегрировать FindBugs в процесс
        сборки на CI и останавливать билд при обнаружении проблем. Результаты
        проверки можно получать либо в виде HTML, либо XML.

        К минусам можно отнести необходимость дополнительной конфигурации и
        сложность интеграции в IDE. Также при использовании Kotlin возможны ложные
        срабатывания в части проверок переменных на \texttt{null}.
        Ещё одним минусом можно назвать то, что выполнение FindBugs требует
        значительных временных и аппаратных затрат. Так как FindBugs анализирует
        байткод, то требуется компиляция всего проекта. Плюс сам процесс проверки
        занимает время сравнимое со сборкой и требует как минимум 1024 Mб оперативной
        памяти. Временные и аппаратные требования могут быть уменьшены за счет
        снижения точности анализа.
    }
\end{frame}

\begin{frame}
    \frametitle{ErrorProne}
    Статический анализатор для поиска потенциальных ошибок в Java-коде на этапе
    компиляции.
    \begin{block}{Возможности}
        \begin{itemize}
            \item{Проверка кода на корректность}
            \item{Проверка на использование известных проблемных подходов}
            \item{Проверка на стилевые несоответствия}
        \end{itemize}
    \end{block}
    \begin{block}{Запуск}
        При подключении Gradle-плагина заменяет собой стандартный компилятор
    \end{block}
    \note{
        Следующий инструмент для статического анализа Java-кода - ErrorProne от
        Google. Как и FindBugs ориентирован на поиск проблем с корректностью и
        использование плохих практик. Но работает по другому. Вместо анализа
        байткода расширяет компилятор и выполняет проверку одновременно с компиляцией.
    }
\end{frame}

\begin{frame}
    \frametitle{ErrorProne}
    \begin{block}{Плюсы}
        \begin{itemize}
            \item{Простота интеграции в IDE}
            \item{Простота интеграции в автоматизированный процесс сборки}
            \item{Быстрое обнаружение ошибок}
            \item{Наличие проверок для популярных библиотек}
            \item{Возможность автоматического исправления найденных ошибок}
        \end{itemize}
    \end{block}
    \begin{block}{Минусы}
        \begin{itemize}
            \item{Прерывает процесс компиляции на первой найденной ошибке}
        \end{itemize}
    \end{block}
    \note{
        К плюсам ErrorProne можно отнести простоту интеграции в IDE и
        автоматизированный процесс сборки. Так как ErrorProne подменяет собой
        компилятор, то проблемы отображаются как ошибки компиляции.
        Это же позволяет находить ошибки ещё на этапе разработки, а не после
        запуска отдельных проверок.
        Кроме того, ErrorProne поставляется с дополнительными проверками для
        ряда известных библиотек (Dagger, Guava, Mockito).
        Ещё одним плюсом ErrorProne является возможность автоматического исправления
        найденных проблем. Во-первых при нахождении проблемы ErrorProne указывает
        в сообщении об ошибке какие-изменения нужно внести для исправления.
        Во-вторых может генерировать патч содержащий исправления для всех найденных проблем.

        К минусам можно отнести прерывание компиляции при нахождении первой проблемы.
        Например если в проекте есть проблемы в 10 разных файлах, то при первой
        компиляции будет показана только проблема из 1-го файла. После исправления
        этой проблемы и перезапуска будет показана проблема из 2-го файла и т. д.
        В результате, чтобы исправить все 10 ошибок придется запускать компиляцию 10 раз.
    }
\end{frame}

\begin{frame}
    \frametitle{Android Lint}
    Статический анализатор для поиска потенциальных ошибок и проблем в Android-проектах.
    \begin{block}{Возможности}
        \begin{itemize}
            \item{Поиск потенциальных ошибок}
            \item{Поиск проблем производительности}
            \item{Поиск проблем  локализации}
            \item{Поиск неиспользуемых ресурсов}
            \item{и т. д.}
        \end{itemize}
    \end{block}
    \begin{block}{Запуск}
        Стандартный Gradle-таск и/или запуск проверки из IDE.
    \end{block}
    \note{
        Если все инструменты рассматриваемые до этого можно применять в чистых
        Java-проектах, то Android Lint ориентирован на поиск проблем специфичных
        для Android-приложений. Например может находить использование API, которые
        требуют более новой версии Android. Как пример проблем с производительностью
        \texttt{lint} может находить создание новых объектов в методах \texttt{onDraw}
    }
\end{frame}

\begin{frame}
    \frametitle{Android Lint}
    \begin{block}{Плюсы}
        \begin{itemize}
            \item{Простота интеграции в IDE}
            \item{Простота интеграции в автоматизированный процесс сборки}
        \end{itemize}
    \end{block}
    \begin{block}{Минусы}
        \begin{itemize}
            \item{Нет поддержки Kotlin при запуске из командной строки}
        \end{itemize}
    \end{block}
    \note{
        К плюсам Lint можно отнести хорошую интеграцию в IDE и Gradle.
        Из IDE Lint может быть запущен из главного меню \texttt{Analyze | Inspect Code}
        Из командной строки Lint можно запустить используя Gradle-таск \texttt{./gradlew lint}
        При запуске из командной строки также генерируются HTML и XML отчеты.

        К минусам можно отнести отсутствие поддержки Kotlin при запуске из
        командной строки, но в версии 3.1 Android Gradle Plugin планируется
        исправить это.
    }
\end{frame}

\begin{frame}
    \frametitle{Detekt}
    Статический анализатор для Kotlin кода.
    \begin{block}{Возможности}
        \begin{itemize}
            \item{Анализ сложности кода}
            \item{Поиск проблем со стилем кода}
            \item{Обнаружение потенциальных ошибок}
            \item{Обнаружение потенциальных проблем с производительностью}
        \end{itemize}
    \end{block}
    \begin{block}{Запуск}
        Gradle-таск после подключения дополнительного Gradle-плагина.
    \end{block}
    \note{
        Инструментов для статического анализа Kotlin-кода к сожалению не так
        много. Intellij IDEA (и Android Studio) содержат некоторые проверки,
        но как уже было сказано их сложно интегрировать в автоматизированный
        процесс сборки.
        Один из доступных инструментов - Detekt.
        Он позволяет как находить потенциальные ошибки в коде (например
        неверные условия в циклах), так и ошибки в стилевом оформлении (имена
        переменных, методов и т. п.). Плюс может анализировать сложность кода
        (число строк в файле, количество методов, длина методов и т. п.).
    }
\end{frame}

\begin{frame}
    \frametitle{Detekt}
    \begin{block}{Плюсы}
        \begin{itemize}
            \item{Широкие возможности по конфигурации}
            \item{Возможность интеграции в автоматизированный процесс сборки}
        \end{itemize}
    \end{block}
    \begin{block}{Минусы}
        \begin{itemize}
            \item{Сложность интеграции в IDE}
        \end{itemize}
    \end{block}
    \note{
        К плюсам можно отнести возможности по конфигурации и возможность
        интеграции в автоматизированный процесс сборки.
        К минусам можно отнести сложность/невозможность интеграции в IDE.
    }
\end{frame}

\begin{frame}
    \frametitle{StrictMode}
    Системная утилита, для обнаружения проблем в ходе выполнения приложения.
    \begin{block}{Возможности}
        \begin{itemize}
            \item{Обнаружение сетевых запросов в главном потоке}
            \item{Обнаружение чтение/записи на диск в главном потоке}
            \item{Обнаружение http-трафика}
            \item{Обнаружение утечек ресурсов}
        \end{itemize}
    \end{block}
    \note{
        Все предыдущие рассмотренные инструменты являются статическими
        анализаторами, т. е. они не могут анализировать поведение кода
        в ходе выполнения. Однако некоторые проблемы могут быть обнаружены
        только в ходе выполнения приложения. Для нахождения таких проблем
        может использоваться \texttt{StrictMode}
    }
\end{frame}

\begin{frame}[fragile]
    \frametitle{StrictMode}
    \begin{block}{Запуск}
        \begin{minted}[autogobble, fontsize=\tiny]{java}
        @Override
        public void onCreate() {
            super.onCreate();
            if (BuildConfig.DEBUG) {
                setupStrictMode();
                // Workaround for https://issuetracker.google.com/issues/36951662
                new Handler().postAtFrontOfQueue(this::setupStrictMode);
            }
        }

        protected void setupStrictMode() {
                StrictMode.setThreadPolicy(
                        new StrictMode.ThreadPolicy.Builder()
                        .detectNetwork()
                        .detectDiskReads()
                        .detectDiskWrites()
                        .penaltyLog()
                        .build());
                StrictMode.setVmPolicy(new StrictMode.VmPolicy.Builder()
                        .detectLeakedSqlLiteObjects()
                        .detectLeakedClosableObjects()
                        .penaltyLog()
                        .penaltyDeath()
                        .build());
        }
        \end{minted}
    \end{block}
\end{frame}

\begin{frame}[fragile]
    \frametitle{LeakCanary}
    Библиотека для обнаружения утечек памяти.
    \begin{block}{Запуск}
        \begin{minted}[autogobble, fontsize=\tiny]{java}
        @Override
        public void onCreate() {
            super.onCreate();
            if (LeakCanary.isInAnalyzerProcess(this)) {
                return;
            }
            LeakCanary.install(this);
        }
        \end{minted}
    \end{block}
    \note{
        Ещё один полезный инструмент для обнаружения проблем в приложении
        во время выполнения - LeakCanary от Square. Данная библиотека позволяет
        обнаруживать утечки памяти. Например, если в приложении есть статическая
        ссылка на экземпляр активности, то при смене конфигурации данный экземпляр
        активности не будет собран сборщиком мусора, что в свою очередь приведет
        к утечке памяти.
    }
\end{frame}

\begin{frame}
    \frametitle{Ссылки}
    \nocite{*}
    \RaggedRight
    \AtNextBibliography{\scriptsize}
    \printbibliography[heading=none]
\end{frame}

\begin{frame}
    \frametitle{Вопросы?}
\end{frame}

\end{document}
