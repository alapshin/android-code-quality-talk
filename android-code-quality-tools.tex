\documentclass{beamer}

\usetheme{metropolis}
\usefonttheme{serif}
\setbeameroption{hide notes}
\setbeamercolor{note page}{bg=white}
\setbeamercolor{note title}{bg=white}
\setbeamertemplate{bibliography item}{}

\usepackage{fontspec}
\setmainfont{Noto Serif}
\setsansfont{Noto Sans}
\setmonofont{Noto Mono}
\newfontfamily{\cyrillicfont}{Noto Serif}
\newfontfamily{\cyrillicfonttt}{Noto Mono}
\usepackage{microtype}

\usepackage{polyglossia}
\setmainlanguage{english}
\setotherlanguage{russian}

\usepackage{csquotes}
\usepackage{graphicx}
\usepackage{ragged2e}
\usepackage[sorting=none]{biblatex}

% Remove last dot from biblatex entries
\renewcommand{\finentrypunct}{}
% Replace dot between title and url with colon
\renewcommand{\newunitpunct}{\addcolon\space}
% Remove URL prefix from url
\DeclareFieldFormat{url}{\url{#1}}
\addbibresource{android-code-quality-tools.bib}

\usepackage{hyperref}
\hypersetup{breaklinks=true,unicode=true,pdfencoding=auto}

\usepackage{shellesc}
\usepackage[outputdir=build]{minted}

\author{Андрей Лапшин}
\date{2017-12-26}
\title{Инструменты качества кода для Android}


\begin{document}
\begin{frame}
    \titlepage
    \note{
        Всем привет. Меня зовут Андрей Лапшин, я занимаюсь разработкой
        под Android и мой доклад будет посвящен различный инструментам,
        которые упрощают процесс поддержания кода в "качественном" состоянии.
    }
\end{frame}

\begin{frame}
    \frametitle{Зачем?}
    \pause
    \begin{itemize}
        \item{Корректность}
            \pause
        \item{Производительность}
            \pause
        \item{Простота поддержки}
    \end{itemize}
    \note<1>{
        Прежде чем рассматривать конкретные инструменты нам нужно определиться,
        что мы понимаем под "качественным" кодом. На первый взгляд мы можем определить
        "качественный" код, как код который выполняет поставленную задачу без ошибок.
        Однако данное определение оставляет желать лучшего.
    }
    \note<2>{
        Во-первых, как мы можем гарантировать что в приложении нет ошибок?
        Как вариант, мы можем внедрить тестирование (ручное и/или автоматическое)
        Но ручное тестирование и написание тестов требует времени и как правило
        сосредоточено на тестировании бизнес-логики. В тоже время практически
        любом языке программировани/фреймворке есть набор паттернов про которые
        известно, что они могут привести к ошибкам. Например сравнение строк
        Было бы хорошо, чтобы такие ситуации отлавливались автоматический и как можно раньше.
    }
    \note<3>{
        Во-вторых, в этом определении нет ни слова про производительность.
        Возможно, в данный момент производительность приемлемая, но где гарантия
        что она оптимальна и в будущем при появлении новых данных не изменится?
        Опять же мы можем прибегнуть к тестированию, но сталкиваемся c необходимостью
        выделения времени. В тоже время есть известные паттерны, которые ведут
        к снижению производительности, например выделение памяти в \texttt{onDraw}
    }
    \note<4>{
        В-третьих, даже если написанный код корректен и производителен возникает
        вопрос о простоте поддержки этого кода в будущем. Возможно этот код
        написан в одном файле размеров несколько тысяч строк и при этом ещё
        части этого файла используют разные стили оформления в разных частях.
    }
\end{frame}

\begin{frame}
    \frametitle{Android Studio (и другие IDE от JetBrains)}
    \begin{block}{Возможности}
        \begin{itemize}
            \item{Потенциальные ошибки}
            \item{Проблемы производительности}
            \item{Поиск неиспользуемого кода}
            \item{Проверка кода на сложность}
            \item{и т. д.}
        \end{itemize}
    \end{block}
    \begin{block}{Запуск}
        Пункт главного меню \texttt{Analyze | Inspect Code}
    \end{block}
    \note{
        Каждая IDE от JetBrains (Android Studio в том числе) поставляется с набором
        проверок.
    }
\end{frame}
\begin{frame}
    \frametitle{Android Studio (и другие IDE от JetBrains)}
    \begin{block}{Плюсы}
        \begin{itemize}
            \item{Отличная интеграция в IDE}
            \item{Поиск проблем на лету}
            \item{Большой набор проверок для разных языков}
            \item{Возможность автоматического исправления найденных проблем}
        \end{itemize}
    \end{block}
    \begin{block}{Минусы}
        \begin{itemize}
            \item{Cложность интеграции в автоматизированный процесс сборки}
        \end{itemize}
    \end{block}
\end{frame}

\begin{frame}
    \frametitle{Checkstyle}
    Статический анализатор для проверки кода на соответствие определенному стилю.
    \begin{block}{Возможности}
        \begin{itemize}
            \item{Проверка отступов, пробелов}
            \item{Наименование переменных}
            \item{Ограничение на длину строк}
            \item{Определение неиспользуемых импортов}
        \end{itemize}
    \end{block}
\end{frame}

\begin{frame}
    \frametitle{Checkstyle}
    \begin{block}{Плюсы}
        \begin{itemize}
            \item{Скорость проверки}
            \item{Гибкие возможности настройки}
            \item{Возможность интеграции в автоматизированный процесс сборки}
        \end{itemize}
    \end{block}
    \begin{block}{Минусы}
        \begin{itemize}
            \item{Сложность интеграции в IDE}
        \end{itemize}
    \end{block}
\end{frame}

\begin{frame}
    \frametitle{FindBugs/SpotBugs}
    Статический анализатор для поиска потенциальных ошибок и других проблем в
    Java-коде. Анализирует не исходный код, а сгенерированный байткод.
    \begin{block}{Возможности}
        \begin{itemize}
            \item{Проверка кода на корректность}
            \item{Проверка на использование известных проблемных подходов}
            \item{Проверки на производительность}
            \item{и т. д.}
        \end{itemize}
    \end{block}
\end{frame}

\begin{frame}
    \frametitle{FindBugs/SpotBugs}
    \begin{block}{Плюсы}
        \begin{itemize}
            \item{Высокая точность}
            \item{Возможность интеграции в автоматизированный процесс сборки}
        \end{itemize}
    \end{block}
    \begin{block}{Минусы}
        \begin{itemize}
            \item{Сложность интеграции в IDE}
            \item{Возможны ложные срабатывания в Kotlin коде}
            \item{Временные и аппаратные требования}
        \end{itemize}
    \end{block}
\end{frame}

\begin{frame}
    \frametitle{ErrorProne}
    Статический анализатор для поиска потенциальных ошибок в Java-коде на этапе
    компиляции.
    \begin{block}{Возможности}
        \begin{itemize}
            \item{Проверка кода на корректность}
            \item{Проверка на использование известных проблемных подходов}
            \item{Проверка на стилевые несоответствия}
        \end{itemize}
    \end{block}
\end{frame}

\begin{frame}
    \frametitle{ErrorProne}
    \begin{block}{Плюсы}
        \begin{itemize}
            \item{Простота интеграции в IDE}
            \item{Простота интеграции в автоматизированный процесс сборки}
            \item{Минимальное число ложных срабатываний}
            \item{Наличие проверок для популярных библиотек}
            \item{Возможность автоматического исправления найденых ошибок}
        \end{itemize}
    \end{block}
    \begin{block}{Минусы}
        \begin{itemize}
            \item{Прерывает процесс компиляции на первой найденной ошибке}
        \end{itemize}
    \end{block}
\end{frame}

\begin{frame}
    \frametitle{Android Lint}
    Статический анализатор для поиска потенциальных ошибок и проблем в Android-проектах.
    \begin{block}{Возможности}
        \begin{itemize}
            \item{Поиск потенциальных ошибок}
            \item{Поиск проблем производительности}
            \item{Поиск проблем  локализации}
            \item{Поиск неиспользуемых ресурсов}
            \item{и т. д.}
        \end{itemize}
    \end{block}
\end{frame}

\begin{frame}
    \frametitle{Android Lint}
    \begin{block}{Плюсы}
        \begin{itemize}
            \item{Простота интеграции в IDE}
            \item{Простота интеграции в автоматизированный процесс сборки}
        \end{itemize}
    \end{block}
    \begin{block}{Минусы}
        \begin{itemize}
            \item{Нет поддержки Kotlin кода при запуске из командной строки}
        \end{itemize}
    \end{block}
\end{frame}

\begin{frame}
    \frametitle{Detekt}
    Статический анализатор для Kotlin кода.
    \begin{block}{Возможности}
        \begin{itemize}
            \item{Анализ сложности кода}
            \item{Поиск проблем со стилем кода}
            \item{Обнаружение потенциальных ошибок}
            \item{Обнаружение потенциальных проблем с производительностью}
        \end{itemize}
    \end{block}
\end{frame}

\begin{frame}
    \frametitle{Detekt}
    \begin{block}{Плюсы}
        \begin{itemize}
            \item{Широкие возможности по конфигурации}
            \item{Возможность интеграции в автоматизированный процесс сборки}
        \end{itemize}
    \end{block}
    \begin{block}{Минусы}
        \begin{itemize}
            \item{Сложность интеграции в IDE}
        \end{itemize}
    \end{block}
\end{frame}

\begin{frame}
    \frametitle{StrictMode}
    Системный класс, позволяющий отлавливать определенный проблемы в ходе
    выполнения приложения и реагировать на них
    \begin{block}{Возможности}
        \begin{itemize}
            \item{Обнаружение сетевых запросов в главном потоке}
            \item{Обнаружение чтение/записи на диск в главном потоке}
            \item{Обнаружение http-трафика}
            \item{Обнаружение утечек ресурсов}
        \end{itemize}
    \end{block}
\end{frame}

\begin{frame}
    \frametitle{LeakCanary}
    Библиотека для обнаружения утечек памяти.
\end{frame}

\begin{frame}
    \frametitle{Ссылки}
    \nocite{*}
    \RaggedRight
    \AtNextBibliography{\scriptsize}
    \printbibliography[heading=none]
\end{frame}

\end{document}
